%This is a LaTeX template for homework assignments
\documentclass{article}
\usepackage[utf8]{inputenc}
\usepackage{amsmath}
\usepackage{float}
\usepackage{placeins}
\usepackage{graphicx}
\usepackage{multicol}
\usepackage[spanish,es-nodecimaldot]{babel}
\usepackage{parskip}
\usepackage{algorithm}
\usepackage{url}
\usepackage{listings}
\usepackage{algpseudocode}
\usepackage{geometry}
\usepackage{amssymb}
\usepackage{dsfont}
\geometry{ top=17mm, bottom=16mm, left=15mm, right=15mm }
\begin{document}
\title{\vspace{-1cm}Deep Learning}
\author{Miguel Vilchis}
\date{03 de agosto de 2018}
\maketitle
\section{Definiciones generales}
  \begin{itemize}
    \item Recordar que el objetivo es encontrar la función que 
    logra clasificar correctamente cada entrada.
    \item Funciones de activación: Se usan funciones no lineales para
    poder aprender funciones no lineales.
    \item Parametrización: Proceso de definir que parametros son
    necesarios para un modelo dado.
    \item Función de perdida: Cuantifica que tan bien la clase de
    salida se apega a la verdadera clase. (Buscamos minimizar dicha
    función)
    \item Función de Scoring: Esta función acepta los datos como
    entrada y los mapea a una etiqueta de clase. Ejemplo: 
    \[ f(x_i, W, b) = Wx_i+b
    \] 
    Donde: 
    \begin{lstlisting}
    W = [K x D], x_i = [D x 1] y b = [K x 1]
    \end{lstlisting}
    Siendo K el número de clases para clasificar.
    \item Bias: Permite recorrer o transladar nuestra función de
    scoring en una u otra dirección sin modificar la matriz de pesos.
  \end{itemize}
\section{Función de perdida}
  Es común usar funciones \textit{hinge} pero se usa mas 
  \textit{cross-entropy loss} y \textit{softmax} en el contexto de
  deep learning y redes convolucionales.
  \begin{itemize}
      \item Multi-class svm loss \\
       Agregando la columna de bias a la matriz de pesos, sea:
       \[
       s = f(x_i, W) = Wx_i
       \]
	   Dado el punto \textit{i-esimo}, el score predicho de la clase \textit{jth} queda definido como:
	   \[
	   s_j = f(x_i, W)_j
	   \]
	   De aquí obtenemos la función \textit{hinge loss function}
	   (sumando las clasificaciones incorrectas de cada clase y
	   comparandola con la salida de la función score):
	   \[
	   L_i = \sum_{j \neq y_i} max(0, s_j - s_{y_i}+1)
	   \]
	   Mientras que la función \textit{squared hinge loss} (penaliza
	   mas la perdida) se define como:
	   \[
	   L_i = \sum_{j \neq y_i} max(0, s_j - s_{y_i}+1)^2
	   \]
    \item Cross-entropy loss: Siendo softmax una función regresa probabilidades 
    para cada clase mientras que la función  \textit{hinge} nos da el margen. El
    clasificador softmax es la generalización de la forma binaria de la regresión
    logística. $f(x)_i = e^{s_{x_i}}/\sum_{j}e^{x_j}$ y la función de perdida \textit{cross-entropy}
    \[
    L_i = -ln(e^{s_{y_i}}/\sum_{j}e^{y_j})
    \]
    Entonces, la función de perdida debe minimizar el logaritmo negativo de la
    probabilidad de la clase correcta: 
    \[
    L_i = -lnP(Y=y_i|X=x_i) 
    \]
    Donde 
    \[
    P(Y=y_i|X=x_i) = e^{s_{y_i}}/\sum_j e^{s_j}
    \]
    Por lo que la función de perdida \textit{cross-entropy}
    \[
    L = \frac{1}{N} \sum_{i=1}^N L_i
    \]
\end{itemize}
\section{Descenso de gradiente}
Nosotros tratamos las superficies de perdida como convexas incluso si no lo son,
porque hace buen trabajo.
El algoritmo de descenso de gradiente tiene dos veritentes:
\begin{enumerate}
    \item La implementación estandart \textit{vanilla}
    \item La versión optimizada \textit{estocástica}
\end{enumerate}
El pseudocodigo del descenso de gradiente es:
 \begin{lstlisting}
  while True:
   Wgradient = evaluate_gradient(loss, data, W)
   W += -alpha *Wgradient
 \end{lstlisting}
Para el gradiente estocástico, en lugar de calcular el gradiente para todo el conjunto de datos, se hace sobre un sampleo de estos. 
 \begin{lstlisting}
  while True:
   batch = next_training_batch(data, 256)
   Wgradient = evaluate_gradient(loss, data, W)
   W += -alpha *Wgradient
 \end{lstlisting}
\section{Momentum}
Sea $W = W - \alpha \bigtriangledown_{W} f(W)$ el termino V momentum, ponderado por $\gamma$: 
\[
V = \gamma V_{t-1} + \alpha \bigtriangledown_{W} f(W)  \longrightarrow W = W - V_t
\]
Comunmente se usa $\gamma = 0.9$
\section{Regularización}
Es la técnica que asegura que nuestro modelo generalice bien, ayudandonos a controlar la capacidad de nuestro modelo.  Sea la función de perdida \textit{cross-entropy} $L$
\[
L_i = -log(e^{s_{y_i}}/ \sum_je^{s_j}) \longrightarrow L = \frac{1}{N}\sum{i=1}^N L_i
\]
La función de regularización o decaemiento del peso se define como: 
\[
R(W) = \sum_i \sum_j W_{i,j}^2
\]
Actualizando la función de perdida: 
\[
L = \frac{1}{N}\sum{i=1}^N L_i + \lambda R(W)
\]
Entonces: 
\[
W = W - \alpha \bigtriangledown_W f(W) -\lambda R(W)
\]
Tipos de regularización:
\begin{enumerate}
    \item L2 o decaemiento de peso: $R(W)= \sum_i \sum_j W_{i,j}^2$
    \item L1: $R(W)= \sum_i \sum_j |W_{i,j}|$
\end{enumerate}
\end{document}

